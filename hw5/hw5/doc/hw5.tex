\documentclass{./tufte-handout}

\usepackage{amsmath}
\usepackage{graphicx}
\setkeys{Gin}{width=\linewidth,totalheight=\textheight,keepaspectratio}
\graphicspath{{graphics/}}

\title{Statustical Machine Learning}
\author[Aakarsh Nair]{Aakarsh Nair\\aakarsh.nair@student.uni-tuebingen.de  \\Matriculation Number :6546577 }
\date{Due: 1 May 2024} 

% The following package makes prettier tables.  We're all about the bling!
\usepackage{booktabs}

% The units package provides nice, non-stacked fractions and better spacing
% for units.
\usepackage{units}

% The fancyvrb package lets us customize the formatting of verbatim
% environments.  We use a slightly smaller font.
\usepackage{fancyvrb}
\fvset{fontsize=\normalsize}

% Small sections of multiple columns
\usepackage{multicol}

% Provides paragraphs of dummy text
\usepackage{lipsum}

% These commands are used to pretty-print LaTeX commands
\newcommand{\doccmd}[1]{\texttt{\textbackslash#1}}% command name -- adds backslash automatically
\newcommand{\docopt}[1]{\ensuremath{\langle}\textrm{\textit{#1}}\ensuremath{\rangle}}% optional command argument
\newcommand{\docarg}[1]{\textrm{\textit{#1}}}% (required) command argument
\newenvironment{docspec}{\begin{quote}\noindent}{\end{quote}}% command specification environment
\newcommand{\docenv}[1]{\textsf{#1}}% environment name
\newcommand{\docpkg}[1]{\texttt{#1}}% package name
\newcommand{\doccls}[1]{\texttt{#1}}% document class name
\newcommand{\docclsopt}[1]{\texttt{#1}}% document class option name


\begin{document}

\maketitle

\section{Exercise 1: Ridge Regression}

\subsection{(2 points) In the following you have to implement least squares and ridge regression (both L2-loss)}

\begin{enumerate}
    \item \textbf{w = LeastSquares(Designmatrix,Y)}:
    \begin{enumerate}
    \item  input: design matrix Φ ∈ Rn×d and the outputs Y ∈ Rn (column vector)
    \item output: weight vector w of least squares regression as column vector
    \end{enumerate}
    \item  \textbf{w = RidgeRegression(Designmatrix,Y,Lambda)}:
    \begin{enumerate}
        \item input: the design matrix Φ ∈ Rn×d, the outputs Y ∈ Rn (column vector), and the regularization
        parameter $\lambda \in \mathbb{R}^+ := \{x \in \mathbb{R}|x \geq 0\}$.
        \item output: weight vector w of ridge regression as column vector. Use the non-normalized version
        $w = (\phi^T \phi + \lambda \mathbb{1}_d)^{−1}\phi^T Y$ 
    \end{enumerate}
    
    Note that that the regression with L1-loss is already provided in 
        \textbf{L1LossRegression(Designmatrix,Y,Lambda)}
\end{enumerate}

\textbf{Answer:}
\subsection{(1 Point) Let us assume that $d =1$. Write a function Baisis(X, k) }
\bibliography{sample-handout}
\bibliographystyle{plainnat}
\end{document}
